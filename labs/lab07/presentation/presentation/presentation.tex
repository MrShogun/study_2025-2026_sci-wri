\documentclass[aspectratio=169]{beamer}

\usetheme{metropolis}
\title{Лабораторная работа №7: Создание презентаций и постеров в \LaTeX}
\subtitle{Computer Skills for Scientific Writing}
\author{Николаев Дмитрий Иванович, НПМмд--02-24}
\institute{Российский университет дружбы народов имени Патриса Лумумбы}
\date{\today}

% \usepackage{polyglossia}
% \setdefaultlanguage[spelling=modern]{russian}
% \setotherlanguage{english}
%\usepackage{fontspec}
% \setmainfont{Carlito}
% \setsansfont{Carlito}

%\usepackage[utf8]{inputenc}
%\usepackage[T2A]{fontenc}
% Исправленный вызов babel для лучшей совместимости с listings
%\usepackage[shorthands=off, main=russian]{babel}


%\usepackage{graphicx}
%\usepackage{listings}
%\usepackage{xcolor}
%\usepackage{amsmath}
%\usepackage{booktabs} % Для красивых таблиц
%\usepackage{tabularx} % Для таблиц заданной ширины
%\usepackage{siunitx}  % Для выравнивания чисел

% Настройки языка и шрифтов
\usepackage{fontspec}
\usepackage[shorthands=off, main=russian]{babel}

% Необходимые пакеты
\usepackage{graphicx}
\usepackage{listings}
\usepackage{xcolor}
\usepackage{booktabs}


% Настройка оформления кода
\lstdefinestyle{mystyle}{
    backgroundcolor=\color{black!5},
    commentstyle=\color{green!40!black},
    keywordstyle=\color{magenta},
    stringstyle=\color{purple},
    basicstyle=\tiny\ttfamily,
    numbers=left,
    breaklines=true,
    numberstyle=\tiny\color{black!60},
    frame=tb,
    framerule=0pt,
}
\lstset{style=mystyle}

\begin{document}

\frame{\titlepage}

\section{Цели и задачи}

\begin{frame}{Цель работы}
    \begin{block}{Цель}
        Освоить инструменты \LaTeX\ для создания визуальных материалов: презентаций (класс \texttt{beamer}) и научных постеров.
    \end{block}

    \begin{alertblock}{Задачи}
        \begin{itemize}
            \item Воспроизвести примеры создания слайдов: структура, блоки, паузы, оверлеи.
            \item Изучить и сравнить 3 метода создания постеров: \texttt{a0poster}, \texttt{beamerposter}, \texttt{tikzposter}.
            \item Выполнить творческое задание: создать собственную презентацию и постер на выбранную тему.
        \end{itemize}
    \end{alertblock}
\end{frame}

\section{Часть 1: Возможности Beamer}

\begin{frame}[fragile]{Базовая структура и Блоки}
    \begin{columns}[T]
        \begin{column}{0.5\textwidth}
% ВАЖНО: \begin и \end прижаты к левому краю!
\begin{lstlisting}[language=tex, caption=Код слайда с блоком]
\documentclass{beamer}
\usetheme{Warsaw}
\begin{document}
\\begin{frame}{Title}
    Text slide...
    \begin{block}{Important}
        Text inside the block
    \end{block}
\\end{frame}
\end{document}
\end{lstlisting}
        \end{column}
        \begin{column}{0.5\textwidth}
            \textbf{Результат:}\\
            \includegraphics[width=\linewidth,height=0.6\textheight,keepaspectratio]{image/2.png}
        \end{column}
    \end{columns}
\end{frame}

\begin{frame}[fragile]{Динамика: Паузы и Uncover}
    Для последовательного вывода информации используются команды \texttt{\textbackslash pause} и \texttt{\textbackslash uncover}.
    
\begin{lstlisting}[language=tex, caption=Пример uncover в формулах]
\begin{align*}
    f'(x) \uncover<2->{&= g'(x)h(x) + \dots} \\
          \uncover<3->{&= \dots}
\end{align*}
\end{lstlisting}
    
    \begin{center}
        \includegraphics[width=0.6\linewidth]{image/4.png}
    \end{center}
\end{frame}

\section{Часть 2: Сравнение постеров}

\begin{frame}{Обзор методов}
    \begin{table}
        \small
        \centering
        \begin{tabular}{l l l}
            \toprule
            Пакет & Базовый класс & Особенности \\
            \midrule
            \textbf{a0poster} & Article & Ручная верстка (\texttt{minipage}, \texttt{multicol}) \\
            \textbf{beamerposter} & Beamer & Использование тем Beamer, блоки \\
            \textbf{tikzposter} & TikZ & Современный блочный дизайн \\
            \bottomrule
        \end{tabular}
    \end{table}
\end{frame}

\begin{frame}[fragile]{Метод 1: a0poster}
    \begin{columns}[T]
        \begin{column}{0.4\textwidth}
\begin{lstlisting}[language=tex]
\documentclass[a0]{a0poster}
\usepackage{multicol}
\begin{document}
 % Заголовок вручную
 \begin{minipage}{\textwidth}
   \Huge Title
 \end{minipage}
 % Контент в колонках
 \begin{multicols}{2}
   Text content...
   \begin{center}
     \includegraphics{...}
     \captionof{figure}{...}
   \end{center}
 \end{multicols}
\end{document}
\end{lstlisting}
        \end{column}
        \begin{column}{0.6\textwidth}
            \centering
            \includegraphics[width=\linewidth,height=0.7\textheight,keepaspectratio]{image/6.png}
        \end{column}
    \end{columns}
\end{frame}

\begin{frame}[fragile]{Метод 2: beamerposter}
    \begin{columns}[T]
        \begin{column}{0.4\textwidth}
\begin{lstlisting}[language=tex]
\documentclass{beamer}
\usepackage[scale=1.4]{beamerposter}
\usetheme{Berlin}

\begin{document}
\\begin{frame}
  \begin{columns}
    \begin{column}{.48\textwidth}
      \begin{block}{Block 1}
         Text...
      \end{block}
    \end{column}
    \begin{column}{.48\textwidth}
       Text...
    \end{column}
  \end{columns}
\\end{frame}
\end{document}
\end{lstlisting}
        \end{column}
        \begin{column}{0.6\textwidth}
            \centering
            \includegraphics[width=\linewidth,height=0.7\textheight,keepaspectratio]{image/7.png}
        \end{column}
    \end{columns}
\end{frame}

\begin{frame}[fragile]{Метод 3: tikzposter}
    \begin{columns}[T]
        \begin{column}{0.4\textwidth}
\begin{lstlisting}[language=tex]
\documentclass{tikzposter}
\usetheme{Desert}
\title{Poster Title}

\begin{document}
 \maketitle % Авто-заголовок
 \block{Introduction}{Text...}
 
 \begin{columns}
  \column{0.5}
  \block{Left}{Content}
  \column{0.5}
  \block{Right}{Content}
 \end{columns}
\end{document}
\end{lstlisting}
        \end{column}
        \begin{column}{0.6\textwidth}
            \centering
            \includegraphics[width=\linewidth,height=0.7\textheight,keepaspectratio]{image/8.png}
        \end{column}
    \end{columns}
\end{frame}

\section{Часть 3: Собственный проект}

\begin{frame}[fragile]{Собственная презентация (BFS)}
    \begin{columns}[T]
        \begin{column}{0.45\textwidth}
            \textbf{Тема:} Алгоритм Поиска в Ширину.
\begin{lstlisting}[language=tex]
\documentclass{beamer}
\usetheme{Madrid}
\begin{document}
\\begin{frame}{Вычислительная сложность}
    Обозначим:
    \begin{itemize}
        \item $|V|$ --- количество вершин.
        \item $|E|$ --- количество ребер.
    \end{itemize}
    
    Сложность алгоритма BFS:
    \begin{align*}
        O(\uncover<2->{|V|} \uncover<3->{+ |E|})
    \end{align*}
    
    \vspace{1cm}
    \uncover<4->{\textbf{Вывод:} Алгоритм линеен относительно размера графа.}
\\end{frame}
\end{document}
\end{lstlisting}
        \end{column}
        \begin{column}{0.55\textwidth}
            \centering
            \includegraphics[width=\linewidth,keepaspectratio]{image/10.png}
        \end{column}
    \end{columns}
\end{frame}

\begin{frame}[fragile]{Собственный постер (tikzposter)}
    \begin{columns}[T]
        \begin{column}{0.4\textwidth}
\begin{lstlisting}[language=tex]
\documentclass[25pt, a0paper, portrait]{tikzposter}
\usetheme{Autumn}
\usecolorstyle{Spain}
\begin{columns}
    \column{0.33}
    \block{Определения}{
        \textbf{\textcolor{red}{Граф}} $G=(V,E)$ состоит из множества вершин $V$ и множества ребер $E$.
        \vspace{1em}
        BFS использует структуру данных \textit{Очередь} (FIFO).
    }
    \column{0.33}
    \block{Визуализация}{
        \begin{center}
            \includegraphics[width=0.8\linewidth]{BFS_Graph.png}
            \captionof{figure}{Пример обхода графа}
        \end{center}
    }
    \column{0.33}
    \block{Сложность}{
        Временная сложность алгоритма составляет:
        \[ O(|V| + |E|) \]
        где $|V|$ --- число вершин, $|E|$ --- число ребер.
        \note[targetoffsetx=0cm, targetoffsety=-2cm, width=6cm]{Линейная сложность!}
    }
\end{columns}
\end{document}
\end{lstlisting}
        \end{column}
        \begin{column}{0.6\textwidth}
            \centering
            \includegraphics[width=\linewidth,height=0.75\textheight,keepaspectratio]{image/11.png}
        \end{column}
    \end{columns}
\end{frame}

\section{Заключение}

\begin{frame}{Выводы}
    \begin{block}{Результаты}
        \begin{itemize}
            \item Получены навыки создания профессиональных презентаций с динамическим контентом в \texttt{beamer}.
            \item Проведено сравнение трех методов верстки постеров. \texttt{tikzposter} признан наиболее эффективным для быстрой визуализации благодаря встроенным темам и блочной структуре.
            \item Подготовлен постер и презентация по теме "Поиск в ширину (BFS)".
        \end{itemize}
    \end{block}
\end{frame}

\end{document}
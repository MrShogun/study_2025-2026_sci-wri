\documentclass{beamer}

% Тема и цветовая схема
\usetheme{Madrid}
\usecolortheme{seahorse}

\usepackage[english,russian]{babel}
\usepackage{amsmath}

\title{Алгоритмы на графах: Поиск в ширину}
\subtitle{Основы алгоритмов}
\author{Дмитрий Николаев}
\institute{РУДН}
\date{\today}

\begin{document}

% Слайд 1: Титульный
\begin{frame}
    \titlepage
\end{frame}

% Слайд 2: Определение (Block)
\begin{frame}{Что такое Граф?}
    \begin{block}{Определение}
        Граф $G$ --- это пара $(V, E)$, где $V$ --- множество вершин, а $E$ --- множество ребер, соединяющих эти вершины.
    \end{block}
    
    \begin{alertblock}{Важно}
        Графы могут быть ориентированными и неориентированными.
    \end{alertblock}
\end{frame}

% Слайд 3: Алгоритм BFS (Pause)
\begin{frame}{Алгоритм поиска в ширину (BFS)}
    Идея алгоритма:
    \begin{enumerate}
        \item Поместить начальную вершину в очередь.
        \pause
        \item Извлечь вершину из очереди и пометить как посещенную.
        \pause
        \item Добавить всех непосещенных соседей в очередь.
        \pause
        \item Повторять, пока очередь не пуста.
    \end{enumerate}
\end{frame}

% Слайд 4: Сложность (Uncover)
\begin{frame}{Вычислительная сложность}
    Обозначим:
    \begin{itemize}
        \item $|V|$ --- количество вершин.
        \item $|E|$ --- количество ребер.
    \end{itemize}
    
    Сложность алгоритма BFS:
    \begin{align*}
        O(\uncover<2->{|V|} \uncover<3->{+ |E|})
    \end{align*}
    
    \vspace{1cm}
    \uncover<4->{\textbf{Вывод:} Алгоритм линеен относительно размера графа.}
\end{frame}

% Слайд 5: Заключение
\begin{frame}{Заключение}
    Поиск в ширину является базовым алгоритмом для:
    \begin{itemize}
        \item Поиска кратчайшего пути в невзвешенном графе.
        \item Поиска компонент связности.
    \end{itemize}
    \vspace{1cm}
    \centering \LARGE Спасибо за внимание!
\end{frame}

\end{document}
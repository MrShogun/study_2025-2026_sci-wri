\documentclass[25pt, a0paper, portrait]{tikzposter}

\usepackage[english,russian]{babel}
\usepackage{caption} % Для captionof
\usepackage{graphicx}

% Настройки темы
\usetheme{Autumn}
\usecolorstyle{Spain}

\title{Алгоритм Поиска в Ширину (BFS)}
\author{Николаев Дмитрий}
\institute{Российский университет дружбы народов}

\begin{document}

\maketitle

% Верхний блок на всю ширину
\block{Введение}{
    Поиск в ширину (Breadth-First Search, BFS) --- один из основных методов обхода графа. Он позволяет найти кратчайший путь в невзвешенном графе.
}

\begin{columns}
    \column{0.33}
    \block{Определения}{
        \textbf{\textcolor{red}{Граф}} $G=(V,E)$ состоит из множества вершин $V$ и множества ребер $E$.
        
        \vspace{1em}
        BFS использует структуру данных \textit{Очередь} (FIFO).
    }
    
    \column{0.33}
    \block{Визуализация}{
        \begin{center}
            \includegraphics[width=0.8\linewidth]{BFS_Graph.png}
            \captionof{figure}{Пример обхода графа}
        \end{center}
    }
    
    \column{0.33}
    \block{Сложность}{
        Временная сложность алгоритма составляет:
        \[ O(|V| + |E|) \]
        где $|V|$ --- число вершин, $|E|$ --- число ребер.
        
        \note[targetoffsetx=0cm, targetoffsety=-2cm, width=6cm]{Линейная сложность!}
    }
\end{columns}

\block{Применение}{
    BFS используется в GPS-навигаторах, социальных сетях (поиск друзей) и сетевых протоколах.
}

\end{document}
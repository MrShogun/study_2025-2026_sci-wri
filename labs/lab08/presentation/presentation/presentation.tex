\documentclass[aspectratio=169]{beamer}

\usetheme{metropolis}
\title{Лабораторная работа №8: Создание диаграмм и рисунков с помощью TikZ в \LaTeX}
\subtitle{Computer Skills for Scientific Writing}
\author{Николаев Дмитрий Иванович, НПМмд--02-24}
\institute{Российский университет дружбы народов имени Патриса Лумумбы}
\date{\today}

%\usepackage{polyglossia}
%\setdefaultlanguage[spelling=modern]{russian}
%\setotherlanguage{english}
\usepackage{fontspec}
%\setmainfont{Carlito}
%\setsansfont{Carlito}

\usepackage[utf8]{inputenc}
\usepackage[T2A]{fontenc}
% Исправленный вызов babel для лучшей совместимости с listings
\usepackage[shorthands=off, main=russian]{babel}


\usepackage{graphicx}
\usepackage{listings}
\usepackage{xcolor}
\usepackage{amsmath}
\usepackage{booktabs} % Для красивых таблиц
\usepackage{tabularx} % Для таблиц заданной ширины
\usepackage{siunitx}  % Для выравнивания чисел

\lstdefinestyle{mystyle}{
    backgroundcolor=\color{black!5},
    commentstyle=\color{green!40!black},
    keywordstyle=\color{blue},
    stringstyle=\color{purple},
    basicstyle=\tiny\ttfamily,
    numbers=left,
    breaklines=true,
    numberstyle=\tiny\color{black!60},
    frame=tb,
    framerule=0pt,
    keepspaces=true,
}
\lstset{style=mystyle}

\begin{document}

\frame{\titlepage}

\section{Цели и задачи}

\begin{frame}{Цель работы}
    \begin{block}{Основная цель}
        Досконально изучить и практически освоить средства создания векторной графики в \LaTeX\ с помощью пакета TikZ.
    \end{block}
    \begin{alertblock}{Задачи}
        \begin{itemize}
            \item Воспроизвести примеры рисования линий, кривых и узлов из пособия.
            \item Освоить построение графиков функций и использование циклов.
            \item Изучить библиотеку \texttt{tikz.math} для построения рекурсивных фракталов.
            \item Выполнить итоговые упражнения: сложный граф, математический график, Ковер Серпинского.
        \end{itemize}
    \end{alertblock}
\end{frame}

\section{Часть 1: Примеры из пособия}

\begin{frame}[fragile]{1.1-1.2: Линии и кривые}
    \begin{columns}[T]
        \begin{column}{0.4\textwidth}
            \begin{block}{Код: Прямые и кривые}
\begin{lstlisting}[language=tex]
\begin{tikzpicture}
  % Ломаная линия
  \draw (-1,0) -- (3,10pt) -- (35:3);
\end{tikzpicture}

\begin{tikzpicture}
  % Стилизация и изгибы
  \draw[->] (-1,0) -| (3,10pt);
  \draw[red] (3,10pt) -- (35:3);
  % Кривые Безье
  \draw[cyan] (-1,0) .. controls (0,-2) .. (5,1);
\end{tikzpicture}

\begin{tikzpicture}
    \draw[dotted,gray] (-1,0) -- (5,1);
    \draw (-1,0) .. controls (0,-2) and (4,2) .. (5,1);
\end{tikzpicture}

\end{lstlisting}
            \end{block}
        \end{column}
        \begin{column}{0.6\textwidth}
            \begin{block}{Результат}
                \centering
                \includegraphics[width=0.7\linewidth]{image/1.png} \\
                \vspace{0.5cm}
                \includegraphics[width=0.9\linewidth]{image/2.png}
            \end{block}
        \end{column}
    \end{columns}
\end{frame}

\begin{frame}[fragile]{1.3-1.4: Узлы (Nodes)}
    \begin{columns}[T]
        \begin{column}{0.45\textwidth}
            \begin{block}{Код: Размещение и оформление}
\begin{lstlisting}[language=tex]
\begin{tikzpicture}[scale=3]
\draw (0,0) -- (1,1) node[midway]{A} node[pos=0.75,above]{B} node[below right]{C};
\end{tikzpicture}
\begin{tikzpicture}[scale=3]
\draw (0,0) to node[midway]{A} node[pos=0.75, above]{B} (1,1) node[right]{C};
\end{tikzpicture}

\begin{tikzpicture}[scale=3]
\draw (0,0) node[circle, draw]{$\sum\limits_{i=1}^{n}n^2$} -- (1,1)
node[rectangle,draw]{$\frac{1}{\sqrt{2}}$};
\end{tikzpicture}
\begin{tikzpicture}[scale=3]
% define nodes
\node[circle,draw] (label1) at (0,0) {$\sum_{i=1}^{n}n^2$};
\node[rectangle,draw] (label2) at (1,1) {$\frac{1}{\sqrt{2}}$};
% draw the line
\draw (label1) -- (label2);
\end{tikzpicture}
\end{lstlisting}
            \end{block}
        \end{column}
        \begin{column}{0.55\textwidth}
            \begin{block}{Результат}
                \centering
                \includegraphics[width=0.9\linewidth]{image/3.png} \\
                \vspace{0.2cm}
                \includegraphics[width=0.9\linewidth]{image/4.png}
            \end{block}
        \end{column}
    \end{columns}
\end{frame}

\begin{frame}[fragile]{1.5: Сложный граф}
    \begin{columns}[T]
        \begin{column}{0.5\textwidth}
            \begin{block}{Код: Именованные узлы}
\begin{lstlisting}[language=tex]
\begin{tikzpicture}[scale=2]
% Define the nodes
\node[circle, draw] at (0,0) (a) {A};
\node[rectangle, fill] at (3,0) (b) {};
\node at (3,0.4) (blabel) {B};
\node[rectangle,rounded corners, draw] at (5,2) (c) {C};
% Draw the paths
\draw[->, green] (a) -- (b) node[midway, below,black]{2};
\draw[<->, blue] (a) to[out=45, in=135] (b);
\draw[->,red] (b)--(c);
\draw[brown,dotted,very thick] (b) |- (c);
\draw[<-,cyan] (b) -| (c);
\draw[thick,black] (a).. controls (1,5) .. (c) node[midway, above]{$\frac{1}{2}$};
\end{tikzpicture}
\end{lstlisting}
            \end{block}
        \end{column}
        \begin{column}{0.5\textwidth}
            \begin{block}{Результат}
                \centering
                \includegraphics[width=0.9\linewidth]{image/5.png}
            \end{block}
        \end{column}
    \end{columns}
\end{frame}

\begin{frame}[fragile]{1.6: Графики и циклы}
    \begin{columns}[T]
        \begin{column}{0.45\textwidth}
            \begin{block}{Код: plot и foreach}
\begin{lstlisting}[language=tex]
\begin{tikzpicture}
\draw [domain=-2:2] plot (\x, {pow(\x,2});
\end{tikzpicture}

\begin{tikzpicture}[scale=1.5]
% Draw the x and y axis, label the axes and the origin
\draw[gray, ->] (-2,0) -- (2,0) node[right]{$x$} node[pos=0.53, below]{$O$};
\draw[gray, ->] (0,-1) -- (0,1.3) node[above]{$y$};
\draw[fill,gray] (0,0) circle [radius=1pt];
% Plot the curve
\draw[blue, thick] [domain=-2:2, samples=150] plot (\x, {cos(pi*\x r)}) node[right]{$y = \cos(x)$};
% Note: the r in the argument of the cosine signifies that we enter \x in radians
\end{tikzpicture}

\begin{tikzpicture}[scale=0.75]
\foreach \x in {0,1,2,3}
\draw[red,thick] (0,\x) circle [radius=\x+1];
\end{tikzpicture}
\end{lstlisting}
            \end{block}
        \end{column}
        \begin{column}{0.55\textwidth}
            \begin{block}{Результат}
                \centering
                \includegraphics[width=1.0\linewidth]{image/6.png}
            \end{block}
        \end{column}
    \end{columns}
\end{frame}

\begin{frame}[fragile]{1.7: Рекурсия (Треугольник Серпинского)}
    \begin{columns}[T]
        \begin{column}{0.5\textwidth}
            \begin{block}{Код: tikzmath}
\begin{lstlisting}[language=tex]
\newcommand\Triangle[2]{
\draw #1 coordinate(a) -- ++(0:#2) coordinate(b);
\draw (a) -- ++(60:#2) coordinate(c);
\fill (a) -- (b) -- (c) -- cycle;
}
\begin{tikzpicture}
\tikzmath{
function sierpinski(\x, \y, \s, \d) {
	if (\d == 0) then {
		{ \Triangle{(\x,\y)}{\s}; };
	} else {
		\u1 = 0.25*\s;
		\u2 = \u1*sqrt(3);
		\u3 = 0.5*\s;
		sierpinski(\x,\y,\u3,\d-1);
		sierpinski(\x+\u3,\y,\u3,\d-1);
		sierpinski(\x+\u1,\y+\u2,\u3,\d-1);
	};
};
\S = 4;
for \d in {0,...,5}{
	\x = (\S+1)*mod(\d,2);
	\y = int(\d/2) * (\S+1);
	sierpinski(\x,-\y,\S,\d);
	};
}
\end{tikzpicture}
\end{lstlisting}
            \end{block}
        \end{column}
        \begin{column}{0.5\textwidth}
            \begin{block}{Результат}
                \centering
                \includegraphics[width=0.75\linewidth]{image/7.png}
            \end{block}
        \end{column}
    \end{columns}
\end{frame}

\section{Часть 2: Итоговые упражнения}

\begin{frame}[fragile]{Упражнение 1: Граф}
    \begin{columns}[T]
        \begin{column}{0.5\textwidth}
\begin{lstlisting}[language=tex]
\begin{tikzpicture}
% Определение узлов в цикле
\foreach \angle/\label/\col in {150/B/white, 270/D/white, 30/F/white} {
\node[circle, draw, double, fill=\col, minimum size=0.8cm] (\label) at (\angle:3cm) {\Large \label};
}
\foreach \angle/\label/\col in {90/A/green, 210/C/green, 330/E/green} {
\node[circle, draw, fill=\col, minimum size=0.8cm] (\label) at (\angle:3cm) {\Large \label};
}
% Соединения
\draw[red] (A) to[out=150, in=40] (B);
\draw[red] (A) to[out=200, in=35] (D) node[midway, right] {\Large $\sqrt{2}$}; % Кривая sqrt(2)
\draw[red] (A) to[out=340, in=90] (E);
\draw[blue, dotted, very thick] (B) -- (F) node[midway, above] {\large 6};
\draw[cyan, thick] (B) to[out=260, in=100] (C);
\draw[blue, dotted] (B) -- (D) node[midway, left] {\large 2};
\draw[blue, dotted] (F) -- (D) node[midway, right] {\large 4};
\draw[cyan, thick] (C) to[out=320, in=125] (D);
\end{tikzpicture}
\end{lstlisting}
        \end{column}
        \begin{column}{0.5\textwidth}
            \begin{block}{Результат}
                \centering
                \includegraphics[width=0.7\linewidth]{image/8.png}
            \end{block}
        \end{column}
    \end{columns}
\end{frame}

\begin{frame}[fragile]{Упражнение 2: Математические графики}
    \begin{columns}[T]
        \begin{column}{0.5\textwidth}
            \begin{block}{Задача и решение}
                Графики $y=e^x$ и $y=\ln(x)$ с осями.
\begin{lstlisting}[language=tex]
\begin{tikzpicture}[scale=1.5]
    % Оси
    \draw[gray, ->] (-1,0) -- (1.75,0) node[right] {\large $x$};
    \draw[gray, ->] (0,-2) -- (0,3.4) node[above] {\large $y$};
    \node[below right, gray] at (0,0) {$O$};
    \draw[fill,gray] (0,0) circle [radius=1pt];
    
    % Метки на осях
    \draw [gray] (1, 2pt) -- (1, -2pt) node[below right = -2.5pt] {\large $x=1$};
    \draw [gray] (2pt, 1) -- (-2pt, 1) node[left] {\large $y=1$};
    
    % График e^x
    \draw[blue, thick, domain=-1:1.2] plot (\x, {exp(\x)}) node[right] {$y=e^x$};
    
    % График ln(x)
    \draw[black, thick, domain=0.1:1.5] plot (\x, {ln(\x)}) node[right] {$y=\ln(x)$};
\end{tikzpicture}
\end{lstlisting}
            \end{block}
        \end{column}
        \begin{column}{0.5\textwidth}
            \begin{block}{Результат}
                \centering
                \includegraphics[width=0.7\linewidth]{image/9.png}
            \end{block}
        \end{column}
    \end{columns}
\end{frame}

\begin{frame}[fragile]{Упражнение 3: Ковер Серпинского}
    \begin{columns}[T]
        \begin{column}{0.55\textwidth}
            %\begin{block}{Алгоритм}
                %Квадрат делится на 9 частей. Центр (1,1) пропускается, для остальных 8 вызывается рекурсия.
\begin{lstlisting}[language=tex]
\tikzmath{
function sierpinski_carpet(\x, \y, \s, \d) {
  if (\d == 0) then {
    { \fill[black] (\x, \y) rectangle (\x+\s, \y+\s); };
  } else {
     \ns = \s/3;
      for \ix in {0, 1, 2} {
        for \iy in {0, 1, 2} {
          if (\ix == 1 && \iy == 1) then {
            % Пропускаем центр (дырка)
           } else {
 sierpinski_carpet(\x + \ix*\ns, \y + \iy*\ns, \ns, \d-1);
            };
          };
        };
      };
  };
}
\tikzmath{ 
\S = 5;
for \d in {1,...,4}{
	\x = (\S+1)*mod(\d-1,2);
	\y = int((\d-1)/2) * (\S+1);
	sierpinski_carpet(\x,-\y,\S,\d);
	};
}
\end{tikzpicture}
\end{lstlisting}
            %\end{block}
        \end{column}
        \begin{column}{0.45\textwidth}
            \begin{block}{Результат (4 итерации)}
                \centering
                \includegraphics[width=0.9\linewidth]{image/10.png}
            \end{block}
        \end{column}
    \end{columns}
\end{frame}

\section{Заключение}

\begin{frame}{Выводы}
    \begin{block}{Результаты работы}
    \begin{itemize}
        \item Изучен синтаксис TikZ: пути, координаты, стили.
        \item Освоена работа с узлами для создания схем и графов.
        \item Реализовано построение точных графиков математических функций.
        \item Применены возможности программирования внутри \LaTeX\ (циклы, переменные, рекурсия) для создания фракталов.
    \end{itemize}
    \end{block}
\end{frame}

\end{document}
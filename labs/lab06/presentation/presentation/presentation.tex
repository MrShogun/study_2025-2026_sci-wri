\documentclass[aspectratio=169]{beamer}

\usetheme{metropolis}
\title{Лабораторная работа №6: Работа с библиографией и цитированием в \LaTeX}
\subtitle{Computer Skills for Scientific Writing}
\author{Николаев Дмитрий Иванович, НПМмд--02-24}
\institute{Российский университет дружбы народов имени Патриса Лумумбы}
\date{\today}

%\usepackage{polyglossia}
%\setdefaultlanguage[spelling=modern]{russian}
%\setotherlanguage{english}
\usepackage{fontspec}
%\setmainfont{Carlito}
%\setsansfont{Carlito}

\usepackage[utf8]{inputenc}
\usepackage[T2A]{fontenc}
% Исправленный вызов babel для лучшей совместимости с listings
\usepackage[shorthands=off, main=russian]{babel}

\usepackage{graphicx}
\usepackage{listings}
\usepackage{xcolor}
\usepackage{booktabs}
\usepackage{amsmath}

% Настройка отображения кода
\lstdefinestyle{mystyle}{
    backgroundcolor=\color{black!5},
    commentstyle=\color{green!40!black},
    keywordstyle=\color{blue},
    stringstyle=\color{purple},
    basicstyle=\tiny\ttfamily,
    numbers=left,
    breaklines=true,
    numberstyle=\tiny\color{black!60},
    frame=tb,
    framerule=0pt,
}
\lstset{style=mystyle}

\begin{document}

\frame{\titlepage}

\section{Введение}

\begin{frame}{Цели и задачи}
    \begin{block}{Цель работы}
        Изучить и практически освоить средства автоматизированной работы с библиографией и цитированием в \LaTeX.
    \end{block}
    \begin{alertblock}{Задачи}
    \begin{itemize}
        \item Создать библиографическую базу данных (\texttt{.bib}).
        \item Освоить рабочий процесс \textbf{BibTeX} с пакетом \texttt{natbib}.
        \item Освоить рабочий процесс \textbf{Biber} с пакетом \texttt{biblatex}.
        \item Научиться создавать гиперссылки в списке литературы.
        \item Выполнить упражнения по добавлению источников и смене стилей цитирования.
    \end{itemize}
    \end{alertblock}
\end{frame}

\section{Часть 1: Примеры из пособия}

\begin{frame}[fragile]{1.1. Создание базы данных (.bib)}
    \begin{columns}[T]
        \begin{column}{0.7\textwidth}
            \textbf{Файл learnlatex.bib} содержит записи о публикациях.
            \begin{block}{Пример записи книги}
\begin{lstlisting}[language=tex]
@article{Thomas2008,
  author = {Thomas, Christine M. and Liu, Tianbiao and Hall, Michael B. and Darensbourg, Marcetta Y.},
  title = {Series of Mixed Valent {Fe(II)Fe(I)} Complexes That Model the {H(OX)} State of [{FeFe}]Hydrogenase: Redox Properties, Density-Functional Theory Investigation, and Reactivity with Extrinsic {CO}},
  journal = {Inorg. Chem.},
  year = {2008},
  volume = {47},
  number = {15},
  pages = {7009-7024},
  doi = {10.1021/ic800654a},
}
@book{Graham1995,
  author = {Ronald L. Graham and 
            Donald E. Knuth and 
            Oren Patashnik},
  title = {Concrete Mathematics},
  publisher = {Addison-Wesley},
  year = {1995},
}
\end{lstlisting}
            \end{block}
        \end{column}
        \begin{column}{0.3\textwidth}
            \vspace{1em}
            Каждая запись имеет уникальный ключ (например, \texttt{Graham1995}), который используется для ссылки в тексте, а также набор полей с информацией об источнике.
        \end{column}
    \end{columns}
\end{frame}

\begin{frame}[fragile]{1.2. BibTeX и пакет natbib}
    \begin{block}{Код (lab6.tex)}
\begin{lstlisting}[language=tex]
\usepackage{natbib}
...
Text citation: \citet{Graham1995}.
Parenthetical: \citep{Thomas2008}.

\bibliographystyle{plainnat}
\bibliography{learnlatex} % без расширения .bib
\end{lstlisting}
    \end{block}
    \begin{alertblock}{Компиляция}
    \texttt{pdflatex} $\to$ \texttt{bibtex} $\to$ \texttt{pdflatex} $\to$ \texttt{pdflatex}
    \end{alertblock}
    \begin{center}
        \includegraphics[width=0.7\textwidth,height=0.25\textheight,keepaspectratio]{image/1.png}
    \end{center}
\end{frame}

\begin{frame}[fragile]{1.3. Biber и пакет biblatex}
    \begin{block}{Код (lab6.tex)}
\begin{lstlisting}[language=tex]
\usepackage[style=authoryear]{biblatex}
\addbibresource{learnlatex.bib} % с расширением .bib
...
Citation: \autocite{Graham1995}.
Text citation: \textcite{Thomas2008}.

\printbibliography
\end{lstlisting}
    \end{block}
    \begin{alertblock}{Компиляция}
    \texttt{pdflatex} $\to$ \texttt{biber} $\to$ \texttt{pdflatex}
    \end{alertblock}
    \begin{center}
        \includegraphics[width=0.7\textwidth,height=0.25\textheight,keepaspectratio]{image/2.png}
    \end{center}
\end{frame}

\section{Часть 2: Итоговые упражнения}

\begin{frame}[fragile]{2.1. Добавление нового источника}
    \begin{columns}[T]
        \begin{column}{0.6\textwidth}
            \begin{block}{Новая запись в .bib}
\begin{lstlisting}
@book{Knuth1984,
    title = {The TeXbook},
    publisher = {Addison-Wesley},
    year = {1984},
    author = {Donald E. Knuth},
    address = {Boston, MA, USA}
}
\end{lstlisting}
            \end{block}
             \begin{block}{Вызов в .tex}
\begin{lstlisting}
\autocite{Knuth1984}
\end{lstlisting}
            \end{block}
\begin{block}{Добавление кликабельных гиперссылок}
\begin{lstlisting}
\usepackage{hyperref}
\end{lstlisting}
\end{block}
        \end{column}
        \begin{column}{0.4\textwidth}
            \textbf{Результат:}
            \vspace{0.5em}
            
            \includegraphics[width=\linewidth]{image/3.png}
        \end{column}
    \end{columns}
\end{frame}

\begin{frame}[fragile]{2.2. Ошибки и стили}
    \begin{columns}[T]
        \begin{column}{0.5\textwidth}
            \textbf{Ссылка на несуществующий ключ:}
            \begin{lstlisting}
\cite{NonExistent}
            \end{lstlisting}
            Приводит к появлению предупреждений (\texttt{undefined citation}) и жирного ключа в PDF (или знаков \textbf{??}).
            
            \vspace{1em}
            \includegraphics[width=0.8\linewidth]{image/4.png}
        \end{column}
        \begin{column}{0.5\textwidth}
            \textbf{Смена стиля на числовой:}
            \begin{lstlisting}
% Для biblatex
\usepackage[style=numeric]{biblatex}
% Для natbib
\usepackage[numbers]{natbib}
            \end{lstlisting}
            Результат: \texttt{[1]} вместо \texttt{(Author, Year)}.
            
            \vspace{0.5em}
            \includegraphics[width=0.9\linewidth]{image/5.png}
        \end{column}
    \end{columns}
\end{frame}

\section{Заключение}

\begin{frame}{Выводы}
    \begin{block}{Результаты работы}
    \begin{itemize}
        \item Успешно воспроизведены примеры работы с библиографией.
        \item Проведено сравнение двух основных подходов: \textbf{BibTeX} (классический, жесткий) и \textbf{biblatex} (современный, гибкий).
        \item Освоена последовательность компиляции, необходимая для корректного отображения ссылок и списка литературы.
        \item Изучены способы настройки стилей цитирования (текстовый, числовой) и добавления гиперссылок.
    \end{itemize}
    \end{block}
\end{frame}



\end{document}
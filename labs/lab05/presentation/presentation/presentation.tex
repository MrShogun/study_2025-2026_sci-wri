\documentclass[aspectratio=169]{beamer}

\usetheme{metropolis}
\title{Лабораторная работа №5: Создание и форматирование таблиц в \LaTeX}
\subtitle{Computer Skills for Scientific Writing}
\author{Николаев Дмитрий Иванович, НПМмд--02-24}
\institute{Российский университет дружбы народов имени Патриса Лумумбы}
\date{\today}

%\usepackage{polyglossia}
%\setdefaultlanguage[spelling=modern]{russian}
%\setotherlanguage{english}
\usepackage{fontspec}
%\setmainfont{Carlito}
%\setsansfont{Carlito}

\usepackage[utf8]{inputenc}
\usepackage[T2A]{fontenc}
% Исправленный вызов babel для лучшей совместимости с listings
\usepackage[shorthands=off, main=russian]{babel}


\usepackage{graphicx}
\usepackage{listings}
\usepackage{xcolor}
\usepackage{amsmath}
\usepackage{booktabs} % Для красивых таблиц
\usepackage{tabularx} % Для таблиц заданной ширины
\usepackage{siunitx}  % Для выравнивания чисел

\lstdefinestyle{mystyle}{
    backgroundcolor=\color{black!5},
    commentstyle=\color{green!40!black},
    keywordstyle=\color{magenta},
    stringstyle=\color{purple},
    basicstyle=\tiny\ttfamily, % Уменьшаем шрифт для слайдов
    numbers=left,
    breaklines=true,
    numberstyle=\tiny\color{black!60},
    frame=tb,
    framerule=0pt,
}
\lstset{style=mystyle}

\begin{document}

\frame{\titlepage}

\section{Цели и задачи}
\begin{frame}{Цель работы}
    \begin{block}{Основная цель}
        Изучить и практически освоить средства создания и форматирования таблиц в \LaTeX.
    \end{block}
    \begin{alertblock}{Ключевые задачи}
    \begin{itemize}
        \item Воспроизвести все примеры из раздела 5 учебного пособия.
        \item Изучить базовые и продвинутые спецификаторы колонок.
        \item Освоить пакет \texttt{booktabs} для создания таблиц профессионального вида.
        \item Научиться объединять ячейки, стилизовать колонки и управлять отступами.
        \item Изучить специализированные пакеты: \texttt{siunitx}, \texttt{tabularx}, \texttt{longtable}, \texttt{threeparttable}.
        \item Выполнить практические упражнения для закрепления материала.
    \end{itemize}
    \end{alertblock}
\end{frame}

\section{Часть 1: Примеры из пособия}

\begin{frame}[fragile]{1.1-1.4: Основы и профессиональное оформление (1/2)}
    \begin{columns}[T]
        \begin{column}{0.45\textwidth}
            \begin{block}{Код: Базовая таблица и \texttt{booktabs}}
            \begin{lstlisting}[language=tex]
% Обычная таблица
\begin{tabular}{lll}
  Animal & Food & Size \\
  ...
\end{tabular}

% Таблица с длинным текстом
\begin{tabular}{cl} ... \end{tabular}
\begin{tabular}{cp{9cm}} ... \end{tabular}

% Сокращённый синтаксис
\begin{tabular}{*{3}{l}}

% Профессиональная таблица
\usepackage{booktabs}
\begin{tabular}{lll}
  \toprule
  Animal & Food & Size \\
  \midrule
  dog & meat & medium \\
  \cmidrule{1-2}
  horse & hay & large \\
  \bottomrule
\end{tabular}
            \end{lstlisting}
            \end{block}
        \end{column}
        \begin{column}{0.55\textwidth}
            \begin{block}{Результат}
            \includegraphics[width=\linewidth,height=0.8\textheight,keepaspectratio]{image/1.png}
            \includegraphics[width=\linewidth,height=0.8\textheight,keepaspectratio]{image/2.png}
            \end{block}
        \end{column}
    \end{columns}
\end{frame}

\begin{frame}[fragile]{1.1-1.4: Основы и профессиональное оформление (2/2)}
	\begin{columns}[T]
		\begin{column}{0.45\textwidth}
       \includegraphics[width=\linewidth,height=1.0\textheight,keepaspectratio]{image/3.png}
		\end{column}
		\begin{column}{0.45\textwidth}
       \includegraphics[width=\linewidth,height=1.0\textheight,keepaspectratio]{image/4.png}
		\end{column}
    \end{columns}
\end{frame}



\begin{frame}[fragile]{1.5-1.6: Объединение ячеек и стилизация колонок}
    \begin{columns}[T]
        \begin{column}{0.45\textwidth}
            \begin{block}{Код: \texttt{\textbackslash multicolumn} и \texttt{>\{\}}}
            \begin{lstlisting}[language=tex]
% Объединение 2-х ячеек
fuath & \multicolumn{2}{c}{unknown} \\
% Переопределение выравнивания в заголовке
\multicolumn{1}{c}{Animal} & \multicolumn{1}{c}{Food} &
\multicolumn{1}{c}{Size}
% Эмуляция верт. объединения
herbivore & horse & large \\
 & deer & medium \\
% Стилизация первой колонки
% курсивом с двоеточием
\begin{tabular}{>{\itshape}l<{:} ll}
\toprule
% Отмена стиля для заголовка
\multicolumn{1}{c}{Animal} & ... \\
\midrule
dog & meat & medium \\
% Удаление отступов с помощью @
\begin{tabular}{l@{ : }l@{\hspace{2cm}}l}
% Добавление отступа и текста с помощью !
\begin{tabular}{l!{:}ll}
% Разная ширина гор. линий
\begin{tabular}{@{} lll@{}} \toprule[2pt]
\end{tabular}
            \end{lstlisting}
            \end{block}
        \end{column}
        \begin{column}{0.55\textwidth}
            \begin{block}{Результат}
                \includegraphics[width=\linewidth,height=0.9\textheight,keepaspectratio]{image/6.png}
                \includegraphics[width=\linewidth,height=0.9\textheight,keepaspectratio]{image/7.png}
            \end{block}
        \end{column}
    \end{columns}
\end{frame}



\begin{frame}[fragile]{1.7: Выравнивание чисел и таблицы фиксированной ширины}
    \begin{columns}[T]
        \begin{column}{0.5\textwidth}
            \begin{block}{Код: \texttt{siunitx}, \texttt{tabular*}, \texttt{tabularx}}
            \begin{lstlisting}[language=tex]
% Выравнивание по десятичному знаку
\usepackage{siunitx}
\begin{tabular}{SS}
\toprule {Values} & {More Values} \\ \midrule
1 & 2.3456 \\ 1.2 & 34.2345 \\
\bottomrule \end{tabular}

% Фикс. ширина (растягивает пробелы)
\begin{tabular*}{0.5\textwidth}
  {@{\extracolsep{\fill}}cc@{}}
  A & B \\ C & D \\
\end{tabular*}

% Фикс. ширина ("резиновая" колонка X)
\usepackage{tabularx}
\begin{tabularx}{0.5\textwidth}{lX}
 A & Long text here...\\
\end{tabularx}
            \end{lstlisting}
            \end{block}
        \end{column}
        \begin{column}{0.5\textwidth}
            \begin{block}{Результат}
                \includegraphics[width=\linewidth,height=0.8\textheight,keepaspectratio]{image/8.png}
            \end{block}
        \end{column}
    \end{columns}
\end{frame}


\begin{frame}[fragile]{1.8-1.9: Многостраничные таблицы, сноски и вложенность}
    \begin{columns}[T]
        \begin{column}{0.5\textwidth}
            \begin{block}{Код: \texttt{longtable}, \texttt{threeparttable}}
            \begin{lstlisting}[language=tex]
% Для таблиц > 1 страницы
\usepackage{longtable}
\begin{longtable}{cc} ... \end{longtable}
% Таблица со сносками
\usepackage{threeparttable}
\begin{table}
\begin{threeparttable}
 \caption{...}
 \begin{tabular}{l}
  Entry\tnote{1} \\
 \end{tabular}
 \begin{tablenotes}
  \item[1] Note text.
 \end{tablenotes}
\end{threeparttable}
\end{table}
% Вложенные таблицы
\begin{tabular}{lc}
 Test & \begin{tabular}{c}A\\a\end{tabular}
\end{tabular}
% Установление дополнительной высоты строк
\begin{center}
\setlength\extrarowheight{2pt}
            \end{lstlisting}
            \end{block}
        \end{column}
        \begin{column}{0.5\textwidth}
            \begin{block}{Результат}
                \includegraphics[width=0.6\linewidth,height=0.9\textheight,keepaspectratio]{image/9.png}
                \includegraphics[width=0.60\linewidth,height=0.9\textheight,keepaspectratio]{image/10.png}
            \end{block}
        \end{column}
    \end{columns}
\end{frame}

\section{Часть 2: Итоговые упражнения}

\begin{frame}[fragile]{Выполнение упражнений (1/2)}
    \begin{alertblock}{Основные наблюдения из экспериментов}
       \begin{itemize}
           \item \textbf{Спецификаторы \texttt{l}, \texttt{c}, \texttt{r}} корректно выравнивают содержимое по левому/правому краю и центру.
           \item \textbf{Меньше ячеек, чем колонок:} \LaTeX\ компилируется без ошибок, оставляя недостающие ячейки пустыми.
           \item \textbf{Больше ячеек, чем колонок:} Компиляция прерывается с ошибкой \texttt{Extra alignment tab...}.
           \item \textbf{\texttt{\textbackslash multicolumn}} эффективно объединяет ячейки и позволяет создавать сложные заголовки.
       \end{itemize}
    \end{alertblock}
\end{frame}


\begin{frame}[fragile]{Выполнение упражнений (2/2)}
    \begin{columns}[T]
	\begin{column}{0.5\textwidth}
		\includegraphics[width=\linewidth,height=0.5\textheight,keepaspectratio]{image/11.png}
		\includegraphics[width=\linewidth,height=0.5\textheight,keepaspectratio]{image/12.png}
	\end{column}
	\begin{column}{0.5\textwidth}
		\begin{block}{Код (\texttt{\textbackslash multicolumn})}
            \begin{lstlisting}[language=tex]
\begin{tabular}{ccc}
  \toprule
  \multicolumn{2}{c}{Merged Header}
    & Single \\
  \cmidrule(r){1-2}
  Sub 1 & Sub 2 & Sub 3 \\
  \midrule
  A & B & C \\
  D & E & F \\
  \bottomrule
\end{tabular}
            \end{lstlisting}
            \end{block}
                \includegraphics[width=\linewidth,height=0.8\textheight,keepaspectratio]{image/13.png}
	\end{column}
    \end{columns}
\end{frame}


\section{Выводы}
\begin{frame}{Выводы}
    \begin{block}{Освоенные навыки}
    \begin{itemize}
        \item Создание таблиц различной сложности, от базовых до многостраничных.
        \item Профессиональное оформление с помощью \texttt{booktabs} (горизонтальные линии, отступы).
        \item Точное выравнивание числовых данных с помощью \texttt{siunitx}.
        \item Верстка таблиц заданной ширины с помощью \texttt{tabular*} и \texttt{tabularx}.
        \item Создание таблиц со сносками (\texttt{threeparttable}).
        \item Применение продвинутых техник: стилизация колонок, вложенные таблицы, управление высотой строк.
    \end{itemize}
    \end{block}
\end{frame}

\end{document}